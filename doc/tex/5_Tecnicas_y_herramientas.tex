\capitulo{5}{Técnicas y herramientas}

Este punto muestra las diferentes técnicas  y herramientas que se han utilizado en el proyecto. Además, también incluye la metodología y gestión seguida en la construcción del proyecto.

\section{Metodología}
Durante el desarrollo del proyecto se ha empleado la metodología CRISP-DM \cite{Haya_2022}, es decir, se han realizado diferentes etapas (entendimiento, preparación, modelado, evaluación y despliegue) con el conjunto de datos original.

Además, en alguna de las etapas se ha empleado la metodología ágil \emph{Scrum}. En dicha metodología \cite{Proyectos_Ágiles_2021} se pretende efectuar las diferentes tareas de manera incremental, formando un \emph{sprint}. 

Finalmente, indicar que se han llevado a cabo reuniones cada dos semanas para revisar el progreso y tomar las decisiones adecuadas en el correcto desarrollo del proyecto.

\section{Herramientas}
\subsection{Anaconda}
\emph{Anaconda}\footnote{Anaconda: \url{https://www.anaconda.com/}} es una distribución de uso libre y abierta basada en \emph{Python} y cuyo principal uso es en las ramas de ciencias de datos y en aprendizaje automático.

\subsection{Jupyter Notebook}
\emph{Jupyter Notebook}\footnote{Jupyter Notebook: \url{https://jupyter.org/}} es una interfaz web utilizada principalmente para ejecutar sentencias de código desde el navegador a través de los \emph{notebooks} que permite generar.

\subsection{GitHub}
\emph{GitHub}\footnote{GitHub: \url{https://github.com/}} es un servicio que permite alojar los proyectos de manera remota, e ir subiendo las nuevas versiones que se desarrollen de manera que se pueda llevar un control de las diferentes versiones, con sus cambios y mejoras.

\section{Bibliotecas de Python}
Durante todo el desarollo del proyecto se ha utilizado el lenguaje de programación \emph{Python} en diferentes \emph{notebooks}. Las bibliotecas que se han empleado durante el desarrollo han sido las siguientes.

\subsection{Pandas}
La librería \emph{Pandas}\footnote{Pandas: \url{https://pandas.pydata.org/}} permite manipular y analizar grandes cantidades de datos. Se asemeja mucho a la herramienta de \emph{Excel} pero en \emph{Python}.

\subsection{NumPy}
\emph{NumPy}\footnote{NumPy: \url{https://numpy.org/}} permite utilizar arrays y matrices sobre las cuales realizar las principales operaciones matemáticas. Además, son muy útilies frente a grandes conjuntos de datos, ya que las operaciones están muy optimizadas.

\subsection{MySQL}
La librería \emph{MySQL}\footnote{MySQL: \url{https://www.mysql.com/}} permite conectarse a una base de datos \emph{MySQL} desde \emph{Python}. Además, permite realizar las principales operaciones en una base de datos tradicional: consultas y añadir o eliminar datos.

\subsection{PyMySQL}
La librería \emph{PyMySQL} es muy similar a la anterior, solo que permite llevar a cabo operaciones en bases de datos de forma mucho más amigable por parte del usuario.

\subsection{TensorFlow}
\emph{TensorFlow}\footnote{TensorFlow:\url{https://www.tensorflow.org}} es una librería desarrollada por \emph{Google} y que se centra principalmente en el uso del aprendizaje automático y la inteligencia artificial. Siendo el primer caso, el utilizado en este proyecto, para poder configurar y crear modelos de aprendizaje automático.

Además, incluye \emph{Keras}\footnote{Keras:\url{https://keras.io/}} lo que simplifica y facilita la construcción de redes neuronales.

\subsection{Scikit-Learn}
La librería \emph{Scikit-learn}\footnote{Scikit-learn:\url{https://scikit-learn.org/}} se centra en el aprendizaje automático, incluyendo los principales algoritmos de clasificación y regresión. Además, incluye alguna herramienta que permite valorar los resultados obtenidos con el modelo, siendo este punto la principal utilidad en el proyecto.

\subsection{Imbalanced-Learn}
\emph{Imbalanced-learn}\footnote{Imbalanced-learn:\url{https://imbalanced-learn.org/}} se centra en abordar el problema de clases desbalanceadas, de modo que tiene múltiples herramientas para intentar de solventar este inconveniente.

\subsection{Matplotlib}
\emph{Matplotlib}\footnote{Matplotlib: \url{https://matplotlib.org/}} permite generar los gráficos más habituales en dos dimensiones, además, estas pueden ser interactivas o dinámicas, facilitando así la comprensión por parte del usuario.

\subsection{Jupyter Widgets}
\emph{Jupyter Widgets}\footnote{Jupyter Widgets: \url{https://github.com/jupyter-widgets/ipywidgets}} otorga a los \emph{notebooks} un cierto dinamismo e interactividad al usuario, de forma que los \emph{notebooks} no queden tan estáticos, y que no suponga únicamente ejecutar celdas sin apenas interacción por parte del usuario.

\section{Documentación}
\subsection{\LaTeX}
\LaTeX\footnote{\LaTeX: \url{https://www.latex-project.org/}} es un sistema de composición de textos usado principalmente en la escritura de textos con alta calidad tipográfica. 

\subsection{Overleaf}
\emph{Overleaf}\footnote{Overleaf: \url{https://www.overleaf.com/}} es un editor colaborativo de \LaTeX{} cuyo principal uso es escribir, editar y publicar documentos científicos, todo ello gestionado desde la nube.

\subsection{Draw.io}
\emph{Draw.io}\footnote{Draw.io: \url{https://app.diagrams.net/}} es una herramienta web que permite, entre otras cosas, crear diagramas, por lo que se ha empleado principalmente para representar pequeños esquemas o diagramas UML.

\subsection{Tables Generator}
\emph{Tables Generator}\footnote{Tables Generator: \url{https://www.tablesgenerator.com/\#}} es una web que facilita la creación de tablas para \LaTeX, ya que se pueden diseñar a través de una interfaz y posteriormente exportarlo a código.

