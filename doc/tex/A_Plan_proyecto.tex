\apendice{Plan de Proyecto Software}

\section{Introducción}
Este primer apéndice contiene información relacionada con la planificación temporal llevada a cabo durante el proyecto y su diferentes viabilidades.

\section{Planificación temporal}
Como ya se ha comentado previamente en la memoria, al hablar de la metodología seguida, la planificación del proyecto se ha basado en la metodología de trabajo \emph{scrum}. Las reuniones se producían, salvo alguna excepción, cada dos semanas donde se producía cada uno de los diferentes \emph{sprints}. En ellos, se revisaban los objetivos marcados en la semana anterior y se marcaban los objetivos del próximo \emph{sprint}.

Además, para la gestión del proyecto y poder reflejar el desarrollo iterativo se ha empleado \emph{GitHub}. El repositorio del proyecto para poder observarlo es el siguiente: \url{https://github.com/ifh1001/TFM}.

\section{Estudio de viabilidad}
Con respecto al estudio de la viabilidad, se va a proceder a analizar la viabilidad económica y legal del proyecto.

\subsection{Viabilidad económica}
Dentro de la viabilidad económica, podemos encontrar dos tipos de gastos:
\begin{itemize}
    \item \textbf{Coste \emph{hardware}}: correspondiente al soporte físico necesario para mantener activa la aplicación.
    \item \textbf{Coste personal}: correspondiente a los contratos de las diferentes personas necesarias para mantener activa la aplicación. 
\end{itemize}
\subsubsection{Coste \emph{hardware}}
La aplicación para poder ejecutarse necesita estar lanzada a través de \emph{Jupyter Notebook}, por lo que los costes necesarios se pueden ver en la Tabla \ref{t:coshar}. En ella se incluyen los elementos básicos necesarios para que un operario pueda utilizar la aplicación.

\begin{center}
\begin{tabular}{|l|l|}
\hline
\textbf{Elemento}    & \textbf{Coste en €} \\ \hline
Ordenador            & 750                 \\ \hline
Monitor              & 100                 \\ \hline
Teclado y ratón      & 20                  \\ \hline
\textbf{Coste total} & \textbf{870}        \\ \hline
\end{tabular}
\captionof{table}{Costes \emph{hardware} estimados}
\label{t:coshar}
\end{center}

\subsubsection{Coste personal}
Para calcular este tipo de gasto, se supone que se contrata a un programador y que en España tienen un salario medio de 29.800€ brutos al año (cantidad obtenida de Jobted\footnote{ Jobted: \url{https://www.jobted.es/salario/programador}}). En la Tabla \ref{t:cosper} puede ver el diferente desglose de gastos según el sueldo bruto estipulado.

\begin{center}
\begin{tabular}{|l|l|}
\hline
\textbf{Concepto}                  & \textbf{Coste en €} \\ \hline
Sueldo neto anual                  & 23.081,5            \\ \hline
Cuota a pagar en el IRPF (16,20\%) & 4.826,2             \\ \hline
Cuotas a la Seguridad Social       & 1.892,3             \\ \hline
\textbf{Sueldo bruto anual}        & \textbf{29.800}     \\ \hline
\end{tabular}
\captionof{table}{Costes personal estimados}
\label{t:cosper}
\end{center}

\subsubsection{Coste total}
Tras haber analizado ambos tipos de costes, en la Tabla \ref{t:costot} se muestran los costes anualizados para contratar a un programador junto con los costes de \emph{hardware} necesarios.

\begin{center}
\begin{tabular}{|l|l|}
\hline
\textbf{Tipo de coste} & \textbf{Coste en €} \\ \hline
Coste hardware         & 870                 \\ \hline
Coste personal         & 29.800              \\ \hline
\textbf{Coste total}   & \textbf{30.670}     \\ \hline
\end{tabular}
\captionof{table}{Costes totales estimados}
\label{t:costot}
\end{center}

\subsection{Viabilidad legal}
El proyecto se ha llevado a cabo en \emph{Python} con diferentes librerías gratuitas, y que todas ellas permiten su comercialización gratuita. 

El problema es que para construir los diferentes modelos en el proyecto, se han empleado datos de bobinas reales y medidas por una empresa, que cómo ya se ha comentado previamente, no tenemos su permiso para mencionar su nombre. Es por ello, que los datos empleados también son confidenciales y no pueden ser divulgados públicamente, lo cual supone un problema si quiere utilizar la aplicación alguien que no sea la empresa, ya que, a priori, no contaría con los datos necesarios para usarse.

Tras todo lo anteriomente comentado, se ha decidido analizar las diferentes licencias asociadas para cada biblioteca y herramienta empleada, y decidir la licencia final del proyecto. Todo esto se puede ver en la Tabla \ref{t:licencias}.

\begin{center}
\begin{tabular}{|l|l|}
\hline
\textbf{Bibliotecas y Herramientas} & \textbf{Licencia} \\ \hline
Jupyter Notebook                    & BSD               \\ \hline
Pandas                              & BSD               \\ \hline
NumPy                               & BSD               \\ \hline
MySQL                               & GNU               \\ \hline
PyMySQL                             & MIT               \\ \hline
TensorFlow                          & Apache 2.0        \\ \hline
Scikit-Image                        & BSD               \\ \hline
Imbalanced-Learn                    & MIT               \\ \hline
Matplotlib                          & PSF               \\ \hline
Jupyter Widgets                     & BSD               \\ \hline
\end{tabular}
\captionof{table}{Licencias Bibliotecas Python y Herramientas}
\label{t:licencias}
\end{center}

Tras analizar las diferentes licencias de las diferentes bibliotecas y herramientas utilizadas, he optado por colocar al proyecto una licencia GPL v3, ya que de esta forma se puede modificar, distribuir, comercializar y realizar patentes del \emph{software} generado durante el desarrollo del proyecto.