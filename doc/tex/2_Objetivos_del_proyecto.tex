\capitulo{2}{Objetivos del proyecto}

Este apartado contiene los diferentes objetivos que se han marcado durante el desarrollo del proyecto. En total se distinguen tres tipos: generales, técnicos y personales. 

\section{Objetivos Generales}
Los objetivos generales marcados en el proyecto son:
\begin{itemize}
    \item Crear atributos que describan la calidad del producto y puedan ser utilizados posteriormente como atributos para modelar.
    \item Evaluar diferentes estrategias de modelado con técnicas de \emph{machine learning} y \emph{deep learning} para realizar recomendaciones.
    \item Realizar una aplicación utilizable por un operario para validar diferentes bobinas.
\end{itemize}

\section{Objetivos Técnicos}
Los objetivos técnicos marcados en el proyecto son:
\begin{itemize}
    \item Desarrollar y entrenar un modelo de \emph{TensorFlow} que sea capaz de identificar el mayor número de bobinas válidas y no válidas para su uso.
    \item Configurar y optimizar los parámetros del modelo con el fin de conseguir la mayor precisión posible.
    \item Preprocesar el conjunto de datos proporcionado por la empresa para que el modelo que se genere pueda identificar de la mejor forma posible las propiedades más relevantes.
\end{itemize}

\section{Objetivos Personales}
Los objetivos personales marcados en el proyecto son:
\begin{itemize}
    \item Aplicar los conocimientos adquiridos a lo largo del máster universitario.
    \item Profundizar, mejorar y aplicar los conocimientos sobre el lenguaje de programación \emph{Python}.
    \item Acercarse a la práctica profesional de las enseñanzas del máster.
\end{itemize}