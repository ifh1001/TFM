\capitulo{2}{Objetivos del proyecto}

Este apartado contiene los diferentes objetivos que se han marcado durante el desarrollo del proyecto. En total se distinguen tres tipos: generales, técnicos y personales. 

\section{Objetivos Generales}
Los objetivos generales marcados en el proyecto son:
\begin{itemize}
    \item Ayudar a la empresa que origina el proyecto, consiguiendo que se puedan identificar el mayor número de bobinas con errores aprovechables por parte de la empresa.
    \item Aumentar la eficiencia de la cadena de galvanizado de la empresa, de manera que un operario no tenga que analizar de forma exhaustiva cada bobina, para saber si es utilizable o no.
    \item Crear una cadena más sostenible, consiguiendo que se puedan aprovechar al máximo las bobinas galvanizadas y evitar que se tenga que desperdiciar demasiado material.    
\end{itemize}

\section{Objetivos Técnicos}
Los objetivos técnicos marcados en el proyecto son:
\begin{itemize}
    \item Desarrollar y entrenar un modelo de \emph{TensorFlow} que sea capaz de identificar el mayor número de bobinas válidas y no válidas para su uso.
    \item Configurar y optimizar los parámetros del modelo con el fin de conseguir la mayor precisión posible.
    \item Preprocesar el conjunto de datos prestado por la empresa para que el modelo que se genere pueda identificar de la mejor forma posible las propiedades más relevantes.
\end{itemize}

\section{Objetivos Personales}
Los objetivos personales marcados en el proyecto son:
\begin{itemize}
    \item Conocer el proceso de galvanizado, con el fin de poder interpretar y desarrollar de la mejora manera posible el proyecto.
    \item Aplicar los conocimientos adquiridos a lo largo del máster universitario.
    \item Profundizar, mejorar y aplicar los conocimientos sobre el lenguaje de programación \emph{Python}.
\end{itemize}