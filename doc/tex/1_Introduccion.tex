\capitulo{1}{Introducción}

El galvanizado \cite{ferrosplanes} es una técnica que se emplea para darle al acero una protección frente a la corrosión, producida principalmente por el aire y la humedad, de forma que estas causan una reacción química o electroquímica, produciendo el deterioro y oxidación del acero. 

Esta técnica consiste en bañar las piezas de acero en zinc, ya que este proporciona al metal, no solo protección frente a la corrosión, sino que también mejora su resistencia a golpes y a la abrasión. Esto permite que este tipo de acero sea idóneo para usarse en el exterior o en lugares con mucha humedad o especialmente corrosivos.

El proceso de galvanizado puede ser de dos tipos:
\begin{itemize}
    \item \textbf{Galvanizado en caliente:} se introduce la pieza de acero en un recipiente que contiene zinc fundido a aproximadamente 450 °C y que se deja hasta que su recubrimiento alcanza las micras deseadas. Generalmente, si se desea que el recubrimiento no dure demasiado tiempo, la capa de zinc tiene un tamaño de entre 7 y 45 micras, y si, en cambio, se desea que el recubrimiento dure más tiempo, el tamaño de la capa de zinc suele estar entre las 45 y las 200 micras.
    \item \textbf{Galvanizado en frío:} para este tipo de galvanizado se aplica zinc mediante pulverización o con electrodeposición hasta que la capa de zinc tiene un tamaño de entre 5 y 20 micras. Este tipo de galvanizado se utiliza generalmente para piezas más pequeñas o incluso para interior, ya que otorga una mejor estética. 
\end{itemize}

Cabe destacar que esta técnica se usa principalmente para la industria de la construcción, y principalmente en aquella dedicada a las piezas metálicas como tuberías o barandillas. Otras de las principales industrias en la que se emplea este tipo de acero es en automoción y en la producción de electrodomésticos. Como se puede ver, las ramas de uso del acero galvanizado son muy variadas, y todo ello debido a sus principales ventajas:
\begin{itemize}
    \item \textbf{Durabilidad y resistencia:} el bañado en zinc del acero aguanta durante una elevada cantidad de años, además de otorgar a la pieza una protección extra frente a los golpes.
    \item \textbf{Protección a la corrosión:} siendo esta una de las principales ventajas y uso de este tipo de acero, ya que evita que la pieza se corroa, alargando la vida útil, soportando incluso los ambientes como mucha humedad y altamente corrosivos.
    \item \textbf{Mantenimiento:} aunque el proceso de galvanizado puede suponer un coste elevado en la fabricación, la durabilidad de este material sin la necesidad de ningún tipo de mantenimiento hace que sea muy cómodo su utilización.
    \item \textbf{Gran versatilidad:} el galvanizado se puede aplicar tanto a piezas grandes, como por ejemplo bobinas con elevadas longitudes de acero, como a piezas pequeñas, como tuercas y tornillos. Además, independientemente de la forma que tenga, se podrá aplicar este proceso, consiguiendo así que el galvanizado se pueda aplicar a multitud de piezas.
    \item \textbf{Fiabilidad:} en la actualidad existen varias leyes que han regulado el acero galvanizado, de manera que las piezas deben de cumplir unos estándares, garantizando así que la pieza cumpla con unos mínimos.
\end{itemize}

Una vez visto que es el galvanizado, como funciona y cuáles son sus principales ventajas, es turno de introducir el origen de este proyecto. Una empresa dedicada a la producción de bobinas galvanizadas, cuyo nombre, por confidencialidad no se puede incluir en esta memoria, nos ha solicitado obtener un modelo que sea capaz de identificar bobinas aprovechables, dentro de aquellas que están descatalogadas por no cumplir con la cantidad de zinc necesaria para garantizar un correcto galvanizado. 

La producción de estas bobinas de acero \cite{gonzalez2008desarrollo} es difícil, ya que siempre el acero tiene que estar entre unos mínimos de zinc, para garantizar un correcto galvanizado y evitar que el material empiece a corroerse antes de tiempo, y entre un máximo, puesto que la empresa estaría perdiendo dinero y además el acero obtenido podría ser demasiado duro impidiendo su uso por parte los clientes. Es por ello, que cuentan con varios sensores que miden la cantidad de zinc que tiene el acero y que identifica aquella bobinas que no cumplen con los criterios de zinc necesarios.

Pese a que haya bobinas que no cumplan con las necesidades de zinc exigidas por el cliente, puede haber ciertas partes aprovechables, por lo que en la actualidad la empresa utiliza un modelo que, aproximadamente, es capaz de identificar correctamente tan solo un 10\% de las bobinas, y el otro 90\% restante depende de un operario que analiza una a una e identifica aquellas que puedan servir. Este proceso es demasiado tedioso y toma mucho tiempo al operario, por lo que se buscará obtener algún modelo que sea capaz de identificar de manera correcta la mayoría de bobinas aprovechables por parte de la empresa, consiguiendo mejorar considerablemente los tiempos de análisis. 

\section{Estructura}
La documentación de este proyecto está formado por un único documento que esta dividido en dos grandes bloques: Memoria y Apéndices, cuya estructura es la siguiente:

\subsection{Memoria}
La memoria está formada por los siguientes puntos:
\begin{enumerate}
    \item \textbf{Introducción:} pone en contexto al lector sobre el problema al que se enfrenta el proyecto. Además, contiene la estructura de toda la documentación.
    \item \textbf{Objetivos del Proyecto:} contiene los objetivos generales, técnicos y personales marcados en el proyecto.
    \item \textbf{Trabajos Relacionados:} habla sobre algunos artículos y trabajos que también están relacionados con el galvanizado y la obtención de modelos basados en el \emph{deep learning}.
    \item \textbf{Conceptos Teóricos:} explicación al lector de los conceptos necesarios para comprender correctamente el proyecto y sus diversas justificaciones.
    \item \textbf{Técnicas y Herramientas:} conjunto de técnicas, metodologías y herramientas que se han utilizado durante el transcurso del proyecto.
    \item \textbf{Aspectos Relevantes del Desarrollo del Proyecto:} describe y explica claramente todo el proceso llevado a cabo durante el proyecto, junto con los problemas y soluciones que se han encontrado.
    \item \textbf{Conclusiones y Líneas de Trabajo Futuras:} tiene la conclusión final del proyecto y como podría seguir evolucionando el proyecto con nuevas mejoras de cara al futuro.
\end{enumerate}

\subsection{Anexo}
El apéndice está formado por un total de cinco puntos:
\begin{enumerate}
    \item \textbf{Plan de Proyecto:} contiene la planificación seguida junto con sus viabilidades económicas y legales.
    \item \textbf{Especificación de Requisitos:} describe los diversos objetivos generales, requisitos y sus correspondientes caso de uso.
    \item \textbf{Especificación de Diseño:} describe los datos, junto con su diseño procedimental y arquitectónico.
    \item \textbf{Documentación Técnica de Programación:} describe la estructura de directorios, el manual del programador, la ejecución del proyecto y las diversas pruebas realizadas al sistema.
    \item \textbf{Documentación de Usuario:} contiene los requisitos de usuario, la instalación y el manual de usuario necesarios para poder utilizar correctamente la aplicación del proyecto.
\end{enumerate}


