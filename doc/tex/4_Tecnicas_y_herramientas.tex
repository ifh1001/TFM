\capitulo{4}{Técnicas y herramientas}

Este punto muestra las diferentes técnicas  y herramientas que se han utilizado en el proyecto. Además, también incluye la metodología y gestión seguida en la construcción del TFM.

\section{Metodología}
Durante el desarrollo del proyecto se ha empleado la metodología ágil del \emph{Scrum}. En dicha metodología se pretende realizar las diferentes tareas de manera incremental, formando un \emph{sprint}. Estos últimos tenían una duración de dos semanas, donde al pasar el tiempo se llevaba a cabo una reunión para revisar el progreso y tomar las decisiones adecuadas en el correcto desarrollo del proyecto.

Con respecto al entendimiento y procesado de los datos se ha seguido la metodología CRISP-DM, es decir, se han realizado diferentes etapas (entendimiento, preparación, modelado, evaluación y despliegue) con el conjunto de datos original.

\section{Gestión del Proyecto}
\subsection{GitHub}
\emph{GitHub}\footnote{GitHub: \url{https://github.com/}} es un servicio que permite alojar los proyectos de manera remota, e ir subiendo las nuevas versiones que se desarrollen de manera que se pueda llevar un control de las diferentes versiones, con sus cambios y mejoras.

\section{Herramientas}
\subsection{Anaconda}
\emph{Anaconda}\footnote{Anaconda: \url{https://www.anaconda.com/}} es una distribución de uso libre y abierta basada en \emph{Python} y cuyo principal uso es en las ramas de ciencias de datos y en aprendizaje automático.

\subsection{Jupyter Notebook}
\emph{Jupyter Notebook}\footnote{Jupyter Notebook: \url{https://jupyter.org/}} es una interfaz web utilizada principalmente para ejecutar sentencias de código desde el navegador a través de los \emph{notebooks} que permite generar.

\section{Bibliotecas de Python}
Durante todo el desarollo del proyecto se ha utilizado el lenguaje de programación \emph{Python} en diferentes \emph{notebooks}. Las bibliotecas que se han empleado durante el desarrollo han sido las siguientes.

\subsection{Pandas}

\subsection{NumPy}

\subsection{MySQL}

\subsection{PyMySQL}

\subsection{TensorFlow}

\subsection{Scikit-learn}

\section{Documentación}
\LaTeX\footnote{\LaTeX: \url{https://www.latex-project.org/}} es un sistema de composición de textos usado principalmente en la escritura de textos con alta calidad tipográfica. 

\subsection{Overleaf}
\emph{Overleaf}\footnote{Overleaf: \url{https://www.overleaf.com/}} es un editor colaborativo de \LaTeX{} cuyo principal uso es escribir, editar y publicar documentos científicos, todo ello gestionado desde la nube.


