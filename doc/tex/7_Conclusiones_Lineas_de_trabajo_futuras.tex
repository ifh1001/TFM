\capitulo{7}{Conclusiones y Líneas de trabajo futuras}

\section{Conclusiones}
En primer lugar, comentar que se han cumplido todos los objetivos generales marcados al inicio del proyecto. A continuación se recogen los diferentes objetivos junto con la explicación de porque se ha cumplido:
\begin{itemize}
    \item \textbf{Crear atributos que describan la calidad del producto y puedan ser utilizados posteriormente como atributos para modelar:} se han modelado diferentes atributos de las bobinas que posteriormente han sido utilizados en un total de 28 experimentos diferentes, en los que se han buscado los mejores modelos que se adaptan al problema del proyecto y a los atributos calculados.
    \item \textbf{Evaluar diferentes estrategias de modelado con técnicas de \emph{machine learning} y \emph{deep learning} para realizar recomendaciones:} dentro de los 28 experimentos se han realizado 14 para datos 1D y otros 14 experimentos para 2D. En dichos experimentos, según el tipo de datos, se han probado diversas estrategias buscando la mejor de ellas.
    \item \textbf{Realizar una aplicación utilizable por un operario para validar diferentes bobinas:} el proyecto va acompañado de un \emph{notebook} que representa la aplicación del proyecto y que simula el proceso que un operario seguiría para llevar a cabo las predicciones sobre unas bobinas en concreto.
\end{itemize}

Pese a estar contentos por cumplir con todos los objetivos generales marcados, no se han obtenido unas precisiones tan buenas como las deseadas. Es por ello, que creemos que al contar con un conjunto de datos desbalanceado y con, tal vez, pocos ejemplos, los resultados no han sido tan buenos como los que cabría esperar. A pesar de ello, pensamos que se han seguido buenas estrategias frente a los datos prestados.

Por otro lado, con los 28 experimentos realizados, se puede sacar en claro que los que mejores resultados han dado, han sido en los casos en los que se han empleado datos 1D de las bobinas, y más en concreto aquellos que utilizan el mapa codificado de las bobinas.

En resumen, se está contento con las diferentes estrategias marcadas y seguidas a lo largo del proyecto, junto con la multitud de diferentes pruebas que se han llevado a cabo. Y aunque los resultados no sean los esperados, se han aprendido multitud de lecciones y conocido un nuevo ámbito laboral relacionado con el máster. 

\section{Líneas de trabajo futuras}
Bajo mi punto de vista, creo que los siguientes pasos y mejoras del proyecto son:
\begin{enumerate}
    \item \textbf{Aumento de bobinas clasificadas pertenecientes a la clase NOK}: con el fin de tener un conjunto de datos más equilibrado y poder crear mejores modelos, sería interesante obtener más bobinas pertenecientes a la clase NOK. De igual manera, si el número de bobinas pertenecientes a la clase OK también aumenta, es bastante posible que el modelo obtenido sea más preciso. En definitiva, cuantos más datos se posean y más equitativo sea el conjunto de datos, los resultados que se consigan serán seguramente mejores.
    \item \textbf{Probar otras alternativas:} es posible que si se emplea algún tipo de red neuronal diferente al empleado en el proyecto, o que si se obtienen nuevas caracteristicas de las bobinas, los modelos muestren mejores resultados. 
    \item \textbf{Desarrollar una aplicación web:} la aplicación actual es un \emph{notebook} sobre el cual se pueden cargar bobinas y con los modelos generados realizar predicciones sobre si serán válidas o no. Pero su interfaz no es del todo amigable, sobre todo para gente que no haya hecho nunca programación y es necesario tener en ejecución el archivo para poder utilizarlo. Es por ello, que sería una gran mejora hacer la aplicación en un entorno web con una interfaz sencilla para que los operarios puedan emplearla fácilmente.
\end{enumerate}
